%-----------------------------------------------------------------------------
%
%               Template for sigplanconf LaTeX Class
%
% Name:         sigplanconf-template.tex
%
% Purpose:      A template for sigplanconf.cls, which is a LaTeX 2e class
%               file for SIGPLAN conference proceedings.
%
% Author:       Paul C. Anagnostopoulos
%               Windfall Software
%               978 371-2316
%               paul@windfall.com
%
% Created:      15 February 2005
%
%-----------------------------------------------------------------------------


\documentclass[preprint]{sigplanconf}

% The following \documentclass options may be useful:
%
% 10pt          To set in 10-point type instead of 9-point.
% 11pt          To set in 11-point type instead of 9-point.
% authoryear    To obtain author/year citation style instead of numeric.

\usepackage{amsmath}
\usepackage[pdftex]{graphicx}

\begin{document}

\conferenceinfo{WXYZ '05}{date, City.}
\copyrightyear{2005}
\copyrightdata{[to be supplied]}

\titlebanner{banner above paper title}        % These are ignored unless
\preprintfooter{short description of paper}   % 'preprint' option specified.

\title{Evaluating Graph Coloring on GPUs}

\authorinfo{Pascal Grosset}
          {University of Utah}
          {pgrosset@sci.utah.edu}
\authorinfo{Peihong Zhu}
          {University of Utah}
          {Email2/3}
\authorinfo{Shusen Liu}
          {University of Utah}
          {Email2/3}
\authorinfo{Mary Hall}
          {University of Utah}
          {mhall@cs.utah.edu}
\authorinfo{Suresh Venkatasubramanian}
          {University of Utah}
          {suresh@cs.utah.edu}

\maketitle

\begin{abstract}
This paper evaluates features of graph coloring algorithms implemented on graphics processing units (GPUs), comparing coloring heuristics, partitioning strategies, and thread decompositions.  As compared to prior work on parallel graph coloring for other parallel architectures, we find that the large number of cores and relatively high global memory bandwidth of a GPU lead to different strategies for the parallel implementation.  Specifically, we find that a simple block partitioning of nodes often outperforms, in terms of execution time or number of colors, the partition found by METIS, which tries to minimize cross-processor edges.  We also discover that our coloring heuristics lead to the same or fewer colors than a prior approach for a distributed-memory cluster architecture.  Our algorithm resolves many coloring conflicts across partitioned blocks on the GPU by iterating through the coloring process up to three times, before returning to the CPU to resolve remaining conflicts, and this iteration dramatically impacts speedups as compared to a single pass.  Overall, parallel graph coloring speeds up well for the sparse graphs we studied, commonly yielding speedups above 1000X over a baseline CPU implementation.
\end{abstract}

\category{CR-number}{subcategory}{third-level}

\terms
term1, term2, term3

\keywords
Graph coloring, Parallel algorithm, CUDA

\section{Introduction}

edges) subject to certain constraints. In this paper, we consider the specific problem of assigning colors to vertices so that no two neighboring vertices (vertices connected by an edge) have the same color.  There are several known applications of graph coloring like assigning frequencies to wireless access points, time-tabling and scheduling, register allocation and printed circuit testing, iterative solution of sparse linear systems [], preconditions [], sparse tiling [], and eigenvalue computation [].  

Theoretically,  graph coloring has been proven to be NP-complete, and even approximating the chromatic number of a graph is known to be NP-hard [reference?]. Therefore, a number of approximate graph coloring algorithms have been developed that use heuristics to guide assignment of colors; some commonly used heuristics include First Fit, Largest Degree Order and Saturation Degree Order.  These heuristics tend to trade off minimizing the number of colors and minimizing execution time.  Combinations of these algorithms have also been used to create better heuristics. These algorithms are generally based on the same general greedy framework: a vertex is selected according to some predefined criteria and colored with the smallest valid color. The selection and coloring continues until all the vertices in the graph are colored.

For coloring large graphs, a parallel implementation seems natural as we can split the graph into small subgraphs, color each of them separately in parallel, and then combine results for all of the subgraphs to derive the final coloring.   Occasionally, this combining step will encounter conflicts, where two adjacent vertices that span multiple subgraphs have been assigned the same color.  A parallel algorithm must resolve conflicts, by assigning one of the vertices involved in the conflict a different color.    Processing conflicts partially serializes the computation and may also lead to the use of a higher number of colors.

This paper examines a graphics processing unit (GPU) mapping of parallel graph coloring.  Prior parallel graph coloring algorithms have been evaluated on conventional shared-memory multiprocessors [reference] or distributed systems [reference], but to the best of our knowledge, this is the first study of how GPU architectures affect performance gains and number of assigned colors in a parallel implementation.  Our study demonstrates that features of the GPU architecture significantly impact the algorithms selected.  Specifically, the support for efficient fine-grain multithreading facilitates strong performance gains over CPU implementations (as high as 1000X) because hundreds or even thousands of threads can be applied to the parallel coloring.  Further, multi-threading can hide the latency to memory of the irregular accesses that on a conventional CPU do not effectively utilize caches.  In addition, most parallel graph algorithms seek to partition graphs carefully to minimize the number of edges that cross processor boundaries, since cross-processor accesses typically have very high communication latencies [need a reference, possibly just a METIS reference].  On a GPU, where the entire graph is stored in a global device memory, the latency of accessing vertices inside or outside the current subgraph is no different, making the partitioning process less important.  \citet{smith02}

Our study also identifies several features of graphs and the GPU implementation that impact both performance and number of assigned colors.  The size and sparsity of a graph affects the number of conflicts, which as suggested above, affects both performance and the quality of the coloring.  Our study evaluates parallelizing polynomial graph coloring algorithms that are known to derive fewer colors than lower complexity algorithms, but for small graphs these more efficient algorithms may also lead to the same quality solution at lower cost.  Therefore, certain graph properties determine whether using a GPU is profitable.  We will evaluate alternative coloring heuristics, partitioning strategies and granularity of the partitioning, to find the solution that leads to a high-quality coloring and a significant performance gain over a baseline CPU implementation.  We compare our coloring results to a prior parallel algorithm on the same set of graphs [from where] [reference?].

The remainder of this paper is structured as follows: we first review previous graph coloring work will be presented.  The following section discusses the details of the parallel algorithm and the coloring heuristics we evaluated.  We present extensive experimental results followed by a conclusion.

\appendix
\section{Appendix Title}

This is the text of the appendix, if you need one.

\acks

Acknowledgments, if needed.

% We recommend abbrvnat bibliography style.

\bibliographystyle{abbrvnat}

% The bibliography should be embedded for final submission.

\begin{thebibliography}{}
\softraggedright

\bibitem[Smith et~al.(2009)Smith, Jones]{smith02}
P. Q. Smith, and X. Y. Jones. ...reference text...




\end{thebibliography}

\end{document}

